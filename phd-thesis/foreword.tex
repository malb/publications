We are standing on the shoulders of giants, so far so general. However, the relative novelty of the fields of cryptography and computational mathematics and the large number conferences, workshops and research links in these areas allow us to meet some of these giants in our fields. In the past three years I had the pleasure of meeting many top researchers in my fields and was also given the opportunity to discuss my often naive problems with them. For this, I am deeply grateful. Not only because these researchers patiently corrected my many mistakes but also because the ECrypt~2 EU project and the Information Security Group funded, supported and thus made possible my travels to workshops, conferences, meetings and other departments. I doubt my research would have turned out the way it did without these lively contacts with the wider research community. 

In a similar fashion the Sage mathematics software project should be mentioned. This project, under the guidance of William Stein, exposed me to a much wider research community than I would have engaged with otherwise. Especially the first part of this thesis is witness to the productive collaborations which arose from this exposure. However, the Sage signature can be found in any given chapter of this thesis. It is not only the direct production of research papers that I took from the Sage project. Without this flexible environment many experiments conducted in this thesis -- which are exclusively performed using Sage -- would have taking much longer and possibly some insights would have been missed. This also applies to the generous support given by William Stein and the Information Security Group in the form of computer resources\footnote{William Stein's computer was purchased under National Science Foundation Grant No. DMS-0821725.}. William Stein and the Sage project also inspired through example: in this project high school students write and review code and documentation alongside distinguished professors as equals. The spirit of the Sage community, especially during the first few Sage workshops, had a severe impact on myself as a researcher.

On a quite fundamental level this thesis would not have been possible without the Royal Holloway Valerie Myerscough Scholarship which funded my studies.

I am deeply grateful to my PhD supervisor Carlos Cid for this supervision, advice and key scientific insights guiding me to the completion of this thesis. Whenever I lost track on how to proceed he offered helpful advice. On the other hand, he encouraged me to pursue my own side-projects allowing me to expand my horizon. Carlos also made numerous suggestions on how to improve this thesis and without his input many issues would not have been spotted. All remaining errors are of course my own. I was lucky to have Carlos Cid as my supervisor. I also wish to thank my PhD adviser Kenneth Paterson with whom I collaborated on several occasions during my PhD. He was key in exposing and pushing me to areas of cryptographic research outside of the topics covered in this thesis. To both researchers I am also grateful for giving advice on more general research questions, career decisions and their general support. I also wish to thank Michael Hortmann who supervised my Diplomarbeit on algebraic attacks which was my entry to cryptographic research.

It is only now that I fully realise how much support I have received from my parents over the past decades which allowed me to pursue this research career path. Their strong appreciation for science, enlightenment and knowledge, their fostering of my interests in computers at a young age and their continued support not only during these past three years made this thesis possible.

Many people provided feedback on individual chapters of this thesis or papers on which some of these chapters are based. These are Gregory Bard, Robert Bradshaw, Michael Brickenstein, Tom Boothby, Stanislav Bulygin, Carlos Cid, Thomas Dullien, Christian Eder, Pooya Farshim, Jean-Charles Faugère, William Hart, Georg Lippold, Wael Said Abd Elmageed Mohamed, Sean Murphy, Matt Robshaw, Cl{\'e}ment Pernet, Ludovic Perret, John Perry, Allan Steel and Ralf-Phillip Weinmann. Since some of the chapters in this thesis are revised versions of published articles anonymous referees also provided feedback and helpful discussions. Furthermore, I wish to express my gratitude towards my co-authors from papers which are not included in this thesis and who were not mentioned so far, these are in alphabetic order: Craig Gentry, Shai Halevi, Jonathan Katz and Gaven Watson.

I finish by expressing my gratitude and appreciation to the one person who probably influenced this thesis the most despite not taking a particular strong interest in my research: Silke Jahn. I cannot think of anyone whose opinion I value more.

\begin{flushright}
-- Martin Albrecht
\end{flushright}
